\documentclass{article}
\usepackage{amsmath} 
\usepackage{amssymb}
\begin{document} 
    \begin{titlepage}
    \begin{center}
    \line(1,0){300}\\
    [0.25 in]
    \huge{\bfseries Shihab Muhtasim}\\
    [0.5 cm]
    \textsc{\Large Student ID: 21301610}\\
    \line(1,0){400}\\
    [2 cm]
    \textsc{\LARGE MAT 110}\\
    [0.5 cm]
    \textsc{\LARGE ASSIGNMENT 01}\\
    [0.5 cm]
    \textsc{\LARGE SET 15}\\
    \end{center}
    \end{titlepage}
\begin{newpage}
    \begin{flushright}
    \textsc{Assignment 1}\\
    \textsc{Problem 1}\\
    [1 cm]
    \end{flushright}
\begin{center}
  \textbf{\Large \underline {Ans to the question no 01}}\\
  [1 cm]
\end{center}
\Large {f(x) will be continuous at x=0 if left hand limit = right hand limit,$\lim_{x \to 0}f(x)=f(0)$ and f(0) is defined.\\[0.5cm]
Now, L.H.L=$\lim_{x \to 0^-}f(x)=\lim_{x \to 0^-}x+2=0+2=0$\\[3mm]
R.H.L=$\lim_{x \to 0^+}f(x)=\lim_{x \to 0^+}x^2-1=0-1=-1$\\[3mm]
$\therefore L.H.L \neq R.H.L$\\[3mm]
Hence, the first condition of continuity is not fulfilled. f(x) is not continuous at x=0 }
\end{newpage}
\begin{newpage}
    \begin{flushright}
    \textsc{Assignment 1}\\
    \textsc{Problem 2}\\
    [1 cm]
    \end{flushright}
\begin{center}
  \textbf{\Large \underline {Ans to the question no 02}}\\
  [1 cm]
\end{center}
\Large {$\lim_{x \to 1}f(x)$ will exist if left hand limit = right hand limit, $\lim_{x \to 1}f(x)=f(1)$ and f(1) is defined.\\[0.5cm]}
Now, L.H.L=$\lim_{x \to 1^-}f(x)=\lim_{x \to 1^-}x+5=1+5=6$\\[3mm]
R.H.L=$\lim_{x \to 0^+}f(x)=\lim_{x \to 0^+}x^3-4x=1-4=-3$\\[3mm]
$\therefore L.H.L \neq R.H.L$\\[3mm]
Hence, {$\lim_{x \to 1}f(x)$ does not exist }
\end{newpage}
\begin{newpage}
    \begin{flushright}
    \textsc{Assignment 1}\\
    \textsc{Problem 3}\\
    [1 cm]
    \end{flushright}
\begin{center}
  \textbf{\Large \underline {Ans to the question no 03}}\\
  [1 cm]
\end{center}
\Large {Evaluating,}\\[2mm]
\LARGE {$\frac{d^2}{dx^2}(\ln{(\frac{x^4+2x^3}{x^2+1})})\\[3mm]
\Rightarrow \frac{d}{dx}(\frac{d}{dx}(\ln(\frac{x^4+2x^3}{x^2+1}))\\[3mm]
\Rightarrow \frac{d}{dx}[(\frac{x^2+1}{x^4+2x^3})\cdot\frac{d}{dx}(\frac{x^4+2x^3}{x^2+1})]\\[3mm]
\Rightarrow \frac{d}{dx}[(\frac{x^2+1}{x^4+2x^3})\cdot\frac{(x^2+1)\frac{d}{dx}(x^4+2x^3)-(x^4+2x^3)\frac{d}{dx}(x^2+1)}{(x^2+1)^2}]\\[3mm]
\Rightarrow \frac{d}{dx}[(\frac{x^2+1}{x^4+2x^3})\cdot\frac{(x^2+1)(4x^3+6x^2)-(x^4+2x^3)(2x)}{(x^2+1)^2}]\\[3mm]
\Rightarrow \frac{d}{dx}[\frac{(x^2+1)(4x^3+6x^2)-2x(x^4.+2x^3)}{(x^4+2x^3)(x^2+1)}]\\[3mm]
\Rightarrow \frac{d}{dx}[\frac{4x^3+6x^2}{x^4+2x^3}-\frac{2x}{x^2+1}]\\[3mm]
\Rightarrow \frac{d}{dx}[\frac{4x+6}{x^2+2x}-\frac{2x}{x^2+1}]\\[3mm]
\Rightarrow \frac{d}{dx}(\frac{4x+6}{x^2+2x})-\frac{d}{dx}(\frac{2x}{x^2+1})\\[3mm]
\Rightarrow \frac{(x^2+2x)\frac{d}{dx}(4x+6)-(4x+6)\frac{d}{dx}(x^2+2x)}{(x^2+2x)^2}-\frac{(x^2+1)\frac{d}{dx}(2x)-(2x)\frac{d}{dx}(x^2+1)}{(x^2+1)^2}\\[3mm]
\Rightarrow \frac{(x^2+2x)\cdot4-(4x+6)\cdot(2x+2)}{(x^2+2x)^2}-\frac{(x^2+1)\cdot2-2x\cdot(2x))}{(x^2+1)^2}\\[3mm]
\Rightarrow \frac{4x^2+8x-8x^2-20x-12}{(x^2+2x)^2}-\frac{2x^2+2-4x^2}{(x^2+1)^2}\\[3mm]
\Rightarrow -\frac{4x^2+12x+12}{(x^2+2x)^2}+\frac{2x^2-2}{(x^2+1)^2}\\[3mm]
\Rightarrow \frac{2x^2-2}{(x^2+1)^2}-\frac{4x^2+12x+12}{(x^2+2x)^2}\\[3mm]
\Rightarrow \frac{(2x^2-2)(x^2+2x)^2-(4x^2+12x+12)(x^2+1)^2}{(x^2+1)^2(x^2+2x)^2}\\[3mm]
\Rightarrow \frac{(2x^2-2)(x^4+4x^3+4x^2)-(4x^2+12x+12)(x^4+2x^2+1)}{x^2(x^2+1)^2(x+2)^2}\\[3mm]
\Rightarrow \frac{2x^6+8x^5+8x^4-2x^4-8x^3-8x^2-4x^6-12x^5-12x^4-8x^4-24x^3-24x^2-4x^2-12x-12}{x^2(x^2+1)^2(x+2)^2}\\[3mm]
\Rightarrow \frac{-2x^6-4x^5-14x^4-32x^3-36x^2-12x-12}{x^2(x^2+1)^2(x+2)^2}$}
\end{newpage}
\begin{newpage}
    \begin{flushright}
    \textsc{Assignment 1}\\
    \textsc{Problem 4}\\
    [1 cm]
    \end{flushright}
\begin{center}
  \textbf{\Large \underline {Ans to the question no 04}}\\
  [1 cm]
\end{center}    
\Large {Given,$ (x-y)^2=x+y-1\\[3mm]
\Rightarrow x^2-2xy+y^2=x+y-1\\[3mm]
\Rightarrow \frac{d}{dx}(x^2-2xy+y^2)=\frac{d}{dx}(x+y-1)\\[3mm]
\Rightarrow 2x-2y+2y\frac{dy}{dx}-2x\frac{dy}{dx}=1+\frac{dy}{dx}-0\\[3mm]
\Rightarrow 2y\frac{dy}{dx}-2x\frac{dy}{dx}-\frac{dy}{dx}=1-2x+2y\\[3mm]
\Rightarrow \frac{d}{dx}(2y-2x-1)=2y-2x+1\\[3mm]
\Rightarrow\frac{d}{dx}=\frac{2y-2x+1}{2y-2x-1}\\[4mm]$
$\therefore \frac{d}{dx}=\frac{2y-2x+1}{2y-2x-1} $}
\end{newpage}
\begin{newpage}
    \begin{flushright}
    \textsc{Assignment 1}\\
    \textsc{Problem 5}\\
    [1 cm]
    \end{flushright}
\begin{center}
  \textbf{\Large \underline {Ans to the question no 05}}\\
  [1 cm]
\end{center}    
\Large {We have to evaluate,\\[0.5cm]
$\frac{d}{dx}(\sin^2{\frac{(2x)}{(x+1)})} \\[0.5cm]
Let,\\[0.25cm]
a=\frac{2x}{x+1}\\
\Rightarrow a= 2x(x+1)^-1\\
\Rightarrow \frac{d}{dx}a=\frac{d}{dx} 2x(x+1)^-1\\
\Rightarrow \frac{d}{dx}(a)= 2(x+1)^-1+2x(-1)(x+1)^-2\\
\Rightarrow \frac{d}{dx}a=\frac{2}{x+1}-\frac{2x}{(x+1)^2}\\ 
\Rightarrow \frac{d}{dx}a=\frac{2(x+1)-2x}{(x+1)^2}\\
\Rightarrow \frac{d}{dx}a=\frac{2x+2-2x}{(x+1)^2}\\
\Rightarrow \frac{d}{dx}a=\frac{2}{(x+1)^2}\\[0.75cm]$
Evaluating,\\[0.5cm]
$\frac{d}{dx}(\sin^2{\frac{(2x)}{(x+1)})}\\
= \frac{d}{dx}(\sin^2a)\\
=2\sin{a}\cos{a}\frac{d}{dx}a\\
=\sin{2a}\frac{d}{dx}a\\
=\sin({\frac{2\cdot2x}{x+1}})\cdot\frac{2}{(x+1)^2}\\
=\frac{2\sin({\frac{4x}{x+1}})}{(x+1)^2}\\[0.25mm]$
We get, $\frac{d}{dx}(\sin^2{\frac{(2x)}{(x+1)})}=\frac{2\sin({\frac{4x}{x+1}})}{(x+1)^2}$}
\end{newpage}
\begin{newpage}
    \begin{flushright}
    \textsc{Assignment 1}\\
    \textsc{Problem 6(a)}\\
    [1 cm]
    \end{flushright}
\begin{center}
  \textbf{\Large \underline {Ans to the question no 06 (a)}}\\
  [1 cm]
\end{center}
\Large {Given,\\[3mm]
$ P(t) = \frac{M}{1+Ae^-kt}\\[3mm]
\lim_{t \to \infty}P(t)= \lim_{t \to \infty}\frac{M}{1+Ae^-kt}\\[3mm]
\Rightarrow \lim_{t \to \infty}P(t) = \lim_{t \to \infty}\frac{M}{1+\frac{A}{e^kt}}\\[3mm] 
\Rightarrow \lim_{t \to \infty}P(t) = \frac{M}{1+\frac{A}{e^k\infty}}\\[3mm]  
\Rightarrow \lim_{t \to \infty}P(t) = \frac{M}{1+\frac{A}{\infty}}\\[3mm]
\Rightarrow \lim_{t \to \infty}P(t) = \frac{M}{1}
\Rightarrow \lim_{t \to \infty}P(t) = M $}\\[0.75cm]
The answer M refers to the maximum population size that can be carried and when t tends to infinity it means that the carrying capacity becomes the maximum population size that can be supported. That's why the answer M is to be expected
\end{newpage}
\begin{newpage}
    \begin{flushright}
    \textsc{Assignment 1}\\
    \textsc{Problem 6(b)}\\
    [1 cm]
    \end{flushright}
\begin{center}
  \textbf{\Large \underline {Ans to the question no 06 (b)}}\\
  [1 cm]
\end{center}
\Large {Given,\\[3mm]
$ P(t) = \frac{M}{1+Ae^-kt}\\[3mm]
\lim_{M \to \infty}P(t)= \lim_{M \to \infty}\frac{M}{1+Ae^-kt}\\[3mm]
\Rightarrow \lim_{M \to \infty}P(t) =\lim_{t \to \infty}\frac{M}{1+(\frac{M-P_o}{P_o})e^-kt}\\[3mm] 
\Rightarrow \lim_{M \to \infty}P(t) =\lim_{M \to \infty}
\frac{MP_o}{P_o(1-e^-kt)+Me^-kt}\\[3mm]  
\Rightarrow \lim_{M \to \infty}P(t) =\lim_{M \to \infty} \frac{P_o}{e^-kt}\\[3mm]
\Rightarrow \lim_{M \to \infty}P(t) = P_oe^{kt}\\[0.75cm]$
The result is $P_oe^{kt}$ which is an exponential function}
\end{newpage}
\end{document} 
