\documentclass{article}
\usepackage{amsmath} 
\usepackage{amssymb}
\begin{document} 
    \begin{titlepage}
    \begin{center}
    \line(1,0){300}\\
    [0.25 in]
    \huge{\bfseries Shihab Muhtasim}\\
    [0.5 cm]
    \textsc{\Large Student ID: 21301610}\\
    \line(1,0){400}\\
    [2 cm]
    \textsc{\LARGE MAT 110}\\
    [0.5 cm]
    \textsc{\LARGE ASSIGNMENT 02}\\
    [0.5 cm]
    \textsc{\LARGE SET 13}\\
    \end{center}
    \end{titlepage}
\begin{newpage}
    \begin{flushright}
    \textsc{Assignment 2}\\
    \textsc{Problem 1}\\
    [1 cm]
    \end{flushright}
\begin{center}
  \textbf{\Large \underline {Ans to the question no 01}}\\
  [1 cm]
\end{center}
\Large {Given function, \\[3mm]
$ f(x)=y = 3x^4-5x^3+6x+8 $\\[3mm]
if we differentiate function f(x) we get,\\[3mm]
$ \frac{dy}{dx}= f\prime (x)= 12x^3-15x^2+6$\\[5mm]
The function f(x) at the point x=1 is,\\[3mm]
$ f(1)= 3(1)^4-5(1)^3+6(1)+8\\[3mm]
\Rightarrow f(1)= 12$} \\[3mm]
Again the function $f\prime(x)$ at x=1 is,\\[3mm]
$ f\prime(1)= 12(1)^3-15(1)^2+6\\[3mm]
\Rightarrow f\prime(1)= 3$ \\[5mm]
Now the equation of the tangent line at x= 1 is,\\[3mm]
$f(x)-f(1)=f\prime(1)\cdot (x-1)\\[3mm]
\Rightarrow y-12= 3 \cdot (x-1) \\[3mm]
\Rightarrow y-12= 3x-3\\[3mm]
\Rightarrow y= 3x+9\\[3mm]
\therefore $ the tangent line of y at x=1 is, $ y=3x+9$
\end{newpage}
\begin{newpage}
\begin{flushright}
    \textsc{Assignment 2}\\
    \textsc{Problem 2}\\
    [1 cm]
    \end{flushright}
\begin{center}
  \textbf{\Large \underline {Ans to the question no 02}}\\
  [1 cm]
\end{center}
\Large {Given function,\\[3mm]
$ x=2\cos{\theta},\\[3mm]
y= 3 \sin{\theta}$\\[5mm]
By differentiating these functions by $\theta$ we get,\\[3mm]
$\frac{dy}{d\theta}= 3\cos{\theta}\\[3mm]
\frac{dx}{d\theta}= -2\sin{\theta}$\\[3mm]
Now,\\[3mm]
$\frac{dy}{dx}= \frac{dy}{d\theta} \cdot \frac{d\theta}{dx}\\[3mm] 
\Rightarrow \frac{dy}{dx}= \frac{\frac{dy}{d\theta}}{\frac{dx}{d\theta}} \\[3mm]
\Rightarrow \frac{dy}{dx}= \frac{3\cos{\theta}}{-2\sin{\theta}}\\[3mm]
\Rightarrow \frac{dy}{dx}= -\frac{3}{2}\cot{\theta}\\[3mm]
\therefore \frac{dy}{dx}$ of the parametric equations is, $-\frac{3}{2}\cot{\theta}$}
\end{newpage}
\begin{newpage}
\begin{flushright}
    \textsc{Assignment 2}\\
    \textsc{Problem 3}\\
    [1 cm]
    \end{flushright}
\begin{center}
  \textbf{\Large \underline {Ans to the question no 03}}\\
  [1 cm]
\end{center}
\Large {Given function,\\[3mm]
$ y=\log _{e} (\frac{2x}{x^2+1})\\[3mm]
\Rightarrow y= \ln (\frac{2x}{x^2+1})$\\[5mm]
By differentiating this logarithmic function y we get,\\[3mm]
$ \frac{dy}{dx}=\frac{d}{dx} (\ln (\frac{2x}{x^2+1}))\\[3mm]
\Rightarrow \frac{dy}{dx}=\frac{1}{\frac{2x}{x^2+1}} \cdot \frac{(x^2+1) \frac{d}{dx}(2x)-(2x)\frac{d}{dx}(x^2+1)}{(x^2+1)^2}\\[3mm]
\Rightarrow \frac{dy}{dx}=\frac{(x^2+1)}{2x} \cdot \frac{2(x^2+1)-2x\cdot2x}{(x^2+1)^2}\\[3mm]
\Rightarrow \frac{dy}{dx}=\frac{2x^2+2-4x^2}{2x(x^2+1)}\\[3mm]
\Rightarrow \frac{dy}{dx}=\frac{2-2x^2}{2x(x^2+1)}\\[3mm]
\Rightarrow \frac{dy}{dx}=\frac{2(1-x^2)}{2x(x^2+1)}\\[3mm]
\Rightarrow \frac{dy}{dx}=\frac{1-x^2}{x^3+x}\\[3mm]
\therefore \frac{dy}{dx}=\frac{1-x^2}{x^3+x}$}
\end{newpage}
\begin{newpage}
\begin{flushright}
    \textsc{Assignment 2}\\
    \textsc{Problem 4}\\
    [0.5 cm]
    \end{flushright}
\begin{center}
  \textbf{\Large \underline {Ans to the question no 04}}\\
  [0.5 cm]
\end{center}
\Large {Given function,\\[2mm]
$f(x)=2+x^2$\\[2mm]
Linear approximation L(x) of the function f(x) at x=3,\\[2mm]
$L(x)= f(3)+f\prime(3)(x-3)$\\[2mm]
The value of f(3) is,\\[2mm]
$f(3)=2+3^2\\[2mm]
\Rightarrow f(3)= 2+9\\[2mm]
\Rightarrow f(3)=11$ \\[2mm]
The value of $f\prime(3)$is,\\[2mm]
$f\prime(x)=2x\\[2mm]
\therefore f\prime(3)= 2\cdot 3\\[2mm]
\Rightarrow f\prime(3)= 6$\\[2mm]
Substituting values of f(3) and $f\prime(3)$ in L(x) at x=3,\\[2mm]
$L(x)= 11+6\cdot (x-3)\\[2mm]
\Rightarrow L(x)=11+6x-18\\[2mm]
\Rightarrow L(x)= 6x-7$\\[2mm]
Now,\\[2mm]
$L(3.1)= 6\cdot (3.1)-7\\[2mm]
\Rightarrow L(3.1)= 18.6-7\\[2mm]
\Rightarrow L(3.1)= 11.6$\\[2mm]}
\end{newpage}
\begin{newpage}
\begin{flushright}
    \textsc{Assignment 2}\\
    \textsc{Problem 5}\\
    [0.5 cm]
    \end{flushright}
\begin{center}
  \textbf{\Large \underline {Ans to the question no 05}}\\
  [0.5 cm]
\end{center}
\Large {Given function,\\[2mm]
$f(x)=x^2-5x+6$\\[3mm]
By differentiating both sides we get, \\[2mm]
$\Rightarrow f\prime(x)=2x-5$\\[3mm]
For extreme values of x, \\[2mm]
$ f\prime(x)=0\\[2mm]
\Rightarrow 2x-5=0\\[2mm]
\Rightarrow x= \frac{5}{2}$\\[3mm]
For the range $(-\infty, 2.5),\\[2mm]
f\prime(1)=2-5\\[2mm]
\Rightarrow f\prime(1)= -3\\[2mm]
\therefore  f\prime(1)<0 $\\[3mm]
For the range $(2.5,\infty),\\[2mm]
f\prime(6)= 2\cdot6-5\\[2mm]
\Rightarrow f\prime(6)= 12-5\\[2mm]
\Rightarrow f\prime(6)=7\\[2mm]
\therefore f\prime(6)>0$\\[3mm]
Left side is decreasing and right side is increasing\\[3mm]
$\therefore$ it is a minima.}
\end{newpage}
\begin{newpage}
\begin{flushright}
    \textsc{Assignment 2}\\
    \textsc{Problem 6}\\
    [0.5 cm]
    \end{flushright}
\begin{center}
  \textbf{\Large \underline {Ans to the question no 06}}\\
  [0.5 cm]
\end{center}
\Large {Given, \\[3mm]
The world population in 2000 was $A_{o}= 6.08$ billion.\\[3mm]
The annual increase rate= 1.26 $\%$\\[3mm]
We know,\\[3mm]
$x(t)= A_{o}\cdot e^{kt}\\[3mm]
\Rightarrow x(t)= 6.08 \cdot e^{\frac{1.26}{100}\cdot t}\\[3mm]
\Rightarrow x(t)= 6.08 \cdot e^{0.0126\cdot t}\\[5mm]$
$\therefore$ Function to population growth since 2000 is,\\[3mm]
$x(t)= 6.08 \cdot e^{0.0126\cdot t}$\\[5mm]
40 years from 2000 in 2040 the population will be,\\[3mm]
$x(t)= 6.08 \cdot e^{0.0126\cdot 40}\\[3mm]
\Rightarrow x(t)= 10.064$ billion\\[3mm]}
\end{newpage}
\end{document}